%%%%%%%%%%%%%%%%%%%%%%%%%%%%%%%%%%%%%%%%%
% Short Sectioned Assignment
% LaTeX Template
% Version 1.0 (5/5/12)
%
% This template has been downloaded from:
% http://www.LaTeXTemplates.com
%
% Original author:
% Frits Wenneker (http://www.howtotex.com)
%
% License:
% CC BY-NC-SA 3.0 (http://creativecommons.org/licenses/by-nc-sa/3.0/)
%
%%%%%%%%%%%%%%%%%%%%%%%%%%%%%%%%%%%%%%%%%

%----------------------------------------------------------------------------------------
%	PACKAGES AND OTHER DOCUMENT CONFIGURATIONS
%----------------------------------------------------------------------------------------

\documentclass[paper=a4, fontsize=12pt]{scrartcl} % A4 paper and 11pt font size

\usepackage[T1]{fontenc} % Use 8-bit encoding that has 256 glyphs
\usepackage{fourier} % Use the Adobe Utopia font for the document - comment this line to return to the LaTeX default
\usepackage[english]{babel} % English language/hyphenation
\usepackage{amsmath,amsfonts,amsthm} % Math packages

\usepackage{lipsum} % Used for inserting dummy 'Lorem ipsum' text into the template

\usepackage{sectsty} % Allows customizing section commands
\allsectionsfont{\centering \normalfont\scshape} % Make all sections centered, the default font and small caps

\usepackage{fancyhdr} % Custom headers and footers
\pagestyle{fancyplain} % Makes all pages in the document conform to the custom headers and footers
\fancyhead{} % No page header - if you want one, create it in the same way as the footers below
\fancyfoot[L]{} % Empty left footer
\fancyfoot[C]{} % Empty center footer
\fancyfoot[R]{\thepage} % Page numbering for right footer
\renewcommand{\headrulewidth}{0pt} % Remove header underlines
\renewcommand{\footrulewidth}{0pt} % Remove footer underlines
\setlength{\headheight}{13.6pt} % Customize the height of the header

\numberwithin{equation}{section} % Number equations within sections (i.e. 1.1, 1.2, 2.1, 2.2 instead of 1, 2, 3, 4)
\numberwithin{figure}{section} % Number figures within sections (i.e. 1.1, 1.2, 2.1, 2.2 instead of 1, 2, 3, 4)
\numberwithin{table}{section} % Number tables within sections (i.e. 1.1, 1.2, 2.1, 2.2 instead of 1, 2, 3, 4)

\setlength\parindent{0pt} % Removes all indentation from paragraphs - comment this line for an assignment with lots of text

%----------------------------------------------------------------------------------------
%	TITLE SECTION
%----------------------------------------------------------------------------------------

\newcommand{\horrule}[1]{\rule{\linewidth}{#1}} % Create horizontal rule command with 1 argument of height

\title{	
\normalfont \normalsize 
\textsc{hochschule bonn-rehin-sieg} \\ [25pt] % Your university, school and/or department name(s)
\horrule{0.5pt} \\[0.4cm] % Thin top horizontal rule
\huge SDP PCL report \\ % The assignment title
\horrule{2pt} \\[0.5cm] % Thick bottom horizontal rule
}

\author{Team members: Iuri Araujo, Shehzad Ahmed, Oscar Lima} % Your name

\date{\normalsize\today} % Today's date or a custom date

\begin{document}

\maketitle % Print the title

%----------------------------------------------------------------------------------------
%	PROBLEM 1
%----------------------------------------------------------------------------------------

\section{Project task description}

Our task for this semester was to understand, and apply different components of PCL (point cloud library) as well as the agentification of the used components.
\\[10pt]
Other goals involved in the project were the constructions of the UML diagrams such as class diagram, sequence diagram and state machine.
\\[10pt]
This will be done by designing and constructing a useful program, implementing all this mentioned requirements.
\\[10pt]
Our program contains a menu which presents the feasible actions such as visualise a point cloud, read a pcd file (point cloud file) and filter application, downsampling or passthrough filter.
\\[10pt]
Later on asks the user to select an option so the program can perform the desired action.
\\[10pt]

%----------------------------------------------------------------------------------------
%	PROBLEM 2
%----------------------------------------------------------------------------------------

\section{Short description of the work done so far}

First we installed PCL library \footnote{http://pointclouds.org}, then we went through the tutorials. We developed 8 examples:
\\[10pt]

\begin{itemize}
  \item pcd\_read  Reads pcd files and prints in console x, y, z coordinates of each point
  \item pcd\_write - Generates file\_name.pcd files out of a given pointcloud
  \item pcl\_viewer - Visualize a given pointcloud
  \item donwsampling - Reduces the size of the cloud computing the centroids of a fixed sized cubes
  \item passthrough\_filter - cuts, or takes slices out of the cloud around a given axis
  \item kinect\_io - Visualise live data from microsoft kinect\footnote{http://www.microsoft.com/en-us/kinectforwindows} sensor device
  \item kinect\_snapshot - Visualise live data from kinect and if user press spacebar then write pcd file
\end{itemize}

After that we integrate this examples into a main program as described in the task description.
\\[10pt]
Finally we generated the UML diagrams of the program. (class, sequence and state machine)
\\[10pt]
All the code is public and available trough github website\footnote{https://github.com/mas-group/sdp\_ws2013\_pcl} inside MAS-GROUP repository.
\\[10pt]


%----------------------------------------------------------------------------------------

%----------------------------------------------------------------------------------------
%	PROBLEM 3
%----------------------------------------------------------------------------------------


\section{Effort spent on the project}


\begin{itemize}
  \item Each of us spent 7 hours per week (in average) for the project
  \item Effort was divided into: 
  \item Understanding pcl library
  \item pcl implementation
  \item Bug fixing
  \item Solving github issues
  \item Installing kinect driver openni\footnote{http://www.openni.org}
  \item CMake\footnote{http://www.cmake.org} usage and eclipse\footnote{http://www.eclipse.org} project creation
\end{itemize}

%----------------------------------------------------------------------------------------

%----------------------------------------------------------------------------------------
%	PROBLEM 4
%----------------------------------------------------------------------------------------

\section{Positive experiences that went well in the first phase}

\begin{itemize}
  \item We got  a general overview understanding of the pcl library
  \item We learn how does the downsampling and passtrought filters works
  \item We improve our git\footnote{http://git-scm.com} skills on a 3 member project
  \item We improve our skills in C++ language
  \item We learn CMake basics
  \item We got the first insights in ROS\footnote{http://www.ros.org} plc usage
\end{itemize}

%----------------------------------------------------------------------------------------

%----------------------------------------------------------------------------------------
%	PROBLEM 5
%----------------------------------------------------------------------------------------

\section{A list of items in need of improvement}

\begin{itemize}
  \item There are still some bugs around the program
  \item Agentification must be improved
  \item Diagrams are having errors, we need to correct them
\end{itemize}

%----------------------------------------------------------------------------------------

%----------------------------------------------------------------------------------------
%	PROBLEM 6
%----------------------------------------------------------------------------------------

\section{Goal suggestions for the second phase}

\begin{itemize}
  \item Current program migration to ROS platform
  \item Proper agentification of the program
  \item Segmentation implementation
  \item Outlier filter implementation
\end{itemize}


%----------------------------------------------------------------------------------------




%----------------------------------------------------------------------------------------

\end{document}
